% This is the introduction chapter for my proposal.
% It has not yet been converted into a dissertation intro.


\chapter{Introduction}

Proposal title: "A poststructural perspective on engineering and technology faculty as learners"


\section{Faculty are learners, too}

We often think of engineering and technology faculty as teachers. As facilitators of student learning, they hold certain philosophies about teaching and learning that shape their interactions in the classroom. However, faculty are also learners themselves, students of the practice of “teaching engineering and technology.” As engineering education researchers concerned with research-to-practice transfer, the ways we think about faculty-as-learners likewise shape the ways we try to impact their practice. If we make this thinking more visible within the engineering/technology domain, it may help the work of engineering and technology faculty members, faculty development professionals, and adult learning researchers.

My dissertation draws on the traditions of cognitive apprenticeship and narrative within engineering education research to explore poststructuralism as one possible perspective on faculty-as-learners. Poststructuralism is a paradigm that constantly seeks a “making-strange” and unsettling of habitual narratives. A poststructural view challenges us to remain in the discomfort of liminal (in-between) spaces where everything is constantly troubled and nothing ever really settles. This study demonstrates a concrete method for engaging faculty members in that liminal space. Through it, I anticipate contributing to our ability to articulate and value faculty explorations in chaotic territories such as large-scale curriculum redesigns, new program formation, and other places where valuable growth occurs but is rarely put into words.

\section{How can we think of faculty as learners?}

I will briefly describe several qualities of faculty-as-learners, compare and contrast them to the qualities presented in existing literature, and explore why it may be valuable to make these qualities visible. Note that by comparing and contrasting my approach to other studies, I am not making statements about the relative quality or validity of their work. I am also not saying that these ways of viewing faculty-as-learners are not present in any existing literature. I am simply using differences and similarities to more clearly articulate certain assumptions about faculty-as-learners that are present in this project.

\subsection{They are situated in the community of practice of teaching their discipline.}

The first quality of faculty-as-learners is that they are situated in a community of practice (Wenger, 1999), that of teaching their discipline. By making sense of the activities of other practitioners around them, they develop the skill of reflection-in-action (Scho¨n, 1983) and use this metacognition and self-monitoring to improve their own practice. Existing engineering education change initiatives have used this cognitive apprenticeship (Collins, Brown, & Newman, 1987; Collins, Brown, & Holum, 1991) approach to build Faculty Learning Communities (Cox, 2004) and other communities of practice to encourage rigorous research in engineering education (Streveler, Smith, & Miller, 2005). Understanding faculty as situated and communal learners helps explain the limited success of an “information dissemination” approach (Siddiqui and Adams, 2013) to research-to-practice transfer. If faculty change their teaching practice primarily in response to direct interactions with colleagues (Fincher, Richards, Finlay, Sharp, & Falconer, 2012) and not print materials, it’s no surprise that journal papers continue to go unread (Borrego, Froyd, & Hall, 2010).

\subsection{They are adult learners.}

The second quality of faculty-as-learners is that they are adult learners. They have rich histories of experience to draw upon as well as expectations of agency (Vella, 1997). This quality allows us to portray faculty as narrators, highly capable and interdependent agents who both read and co-author the stories of “how things are done” within their culture. The Disciplinary Commons initiative (Tenenberg & Fincher, 2007), which scaffolds faculty through a dialogic process of creating teaching portfolios for their existing courses, is an exemplar of validating faculty as adult learners. Initiatives that treat faculty as mere empty “buckets” to be filled with “information about teaching,” such as information sessions dedicated to lecturing faculty about why they should not lecture, may have a limited impact because of this epistemological disjoint.

\subsection{They learn by engaging in liminal experiences.}

The third quality of faculty-as-learners is that they often learn by engaging in liminal experiences where their activities do not fall into a cleanly articulable structure, and can therefore be described as poststructural. In fact, the liminal experiences of faculty are often explicitly about dismantling old structures and remaking new ones, such as the founding of a new college (S. Kerns, Miller, & D. Kerns, 2005) or degree program (Katehi et. al., 2004), dramatic overhauls of an existing curriculum (Mentkowski, 2000), or the creation of experimental classes.

The trouble with liminal experiences is that they stand outside the realm of structure, including the structure of validation and reward. Our existing academic system rewards faculty for engaging in liminal spaces for their research, so long as they are able to publish papers about it afterwards. However, the examples listed all impact teaching, a form of scholarship that is underdeveloped and unrecognized (Shulman, 1998), with a correspondingly slow speed of change. Mann’s (1918) calls for engineering educators to improve student retention, make undergraduate workloads reasonable, and increase hands-on training still sound as relevant today as when he wrote his report nearly a century ago. Our calls for transforming engineering education (National Academy of Engineering, 2005; Institute of Medicine, National Academy of Sciences, & National Academy of Engineering, 2007; McKenna, 2011) will similarly go unheeded unless we find ways to understand and articulate the ways faculty learn and work in these liminal spaces.

\section{Research question}

By having engineering and technology faculty work with their stories of liminal experiences, I address the following research question: How can the interacting narratives of engineering and technology faculty inform our understanding of faculty as learners?
