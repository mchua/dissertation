\section{Considerations and conclusion}

This section begins by going through study logistics for planned outputs and timelines. We then move into a discussion on various facets of poststructural validity, which differ significantly from conventional positivist notions of validity. We discuss several concerns and considerations regarding project design and execution, including the effects of prior relationships with faculty narrators, decentering researcher privilege, power dynamics within the study, and the effect of my deafness on data quality and the overall study design. We end with a conclusion summarizing the entire proposal document.

\subsection{Logistics}

This subsection describes the ouputs that the study will produce and explains the timeline decisions behind producing them.

\subsubsection{Outputs produced}

This study must adhere to at least some cultural norms of engineering education research in order to present itself within that culture. Accordingly, one of the study outputs will be an extensive single-author dissertation document depicting one researcher’s limited viewpoint of the multivocal project. In addition to this document, the following artifacts will be produced in a public and open-licensed dataset:

A narrative accrual, sub-dividable into 36 realtime transcripts of narrative elicitation sessions with design thinking curriculum revisions (DTCRs) as a point of commonality.
A “cast of characters” portrayed within the narrative accrual, consisting of a series of intertextual character sketches.
A collection of rhizomatic fragments woven from and within the narrative accrual.

\subsubsection{Timeline considerations}

As of this writing, I have received IRB approval, recruited all 6 faculty narrators, and begun conducting narrative elicitation sessions based on when narrators were available. The Berea faculty (Alan, Gary, and Mark) have all completed their first sessions; of the Olin faculty, Rob and Jon have each completed 3 sessions and Lynn has completed 2. Because of this, I am also already writing rhizomatic fragments to serve as narrative session prompts.

The onset of data collection should not be taken as an indication that this proposal is set in stone; the analysis plan is extremely revisable, and data collected thus far can always be treated as “pilot” data and re-collected if the design of narrative elicitation sessions needs to be seriously reconsidered.

If the existing data from narrative sessions is workable, it may be possible for me to conduct the remaining narrative sessions over the summer, especially since I plan to visit both Berea and Olin in person multiple times between May-August 2014. Even if a few narrative sessions remain to be conducted by the start of the school year, it is highly probable that they will all be complete by the end of fall semester 2014, allowing me to devote most of fall 2014 and all of spring 2015 to analysis and writing with the target of a spring or summer 2015 defense.

\subsection{Validity}

This subsection summarizes the study’s approach to validity from a poststructural perspective. It describes several facets of poststructural validity adopted from Patti Lather’s various works and explains how this particular study addresses each facet.

\subsubsection{Triangulation: drawing from multiple sources}

Lather’s 1986 paper, “Issues of validity in openly ideological research: Between a rock and a soft place,” lays out 4 guidelines (p. 67) that appear repeatedly in her later work. The first guideline is triangulation, which in this setting means drawing from “multiple data sources, methods, and theoretical schemes.” This study has multiple data sources by virtue of its 6 faculty narrators and an open-licensed narrative accrual that admits community analysis as data. For example, I brought a public transcript to my March 2014 guest lecture in Lather’s advanced qualitative methods class and had her students analyze it. Their notes, which illuminated some tradeoffs between text and speech communication I hadn’t previously considered, are now part of the dataset. This open-ended community involvement and the technique of grounded indigenous coding admit a variety of data collection and analysis methods to the study. Cognitive apprenticeship, narrative analysis, and poststructuralism are the three most visible theoretical schemes woven into its rhizomatic fragments.

\subsubsection{Construct and face validities: narrator reflexivity and affirmation}

The second guideline is construct validity, a systemized reflexivity regarding the theories we operate within, our preconceptions about them, and how we critique and transform them in response to our actual data. I have begun to exhibit construct validity in this document, primarily through placing multiple theoretical frameworks in dialogue with each other and showing how they are both reflected in and modified by my pilot data. 

The third guideline, face validity, is addressed through grounded indigenous coding. Face validity is granted when study participants consider the study’s analysis to be valid. Grounded indigenous coding directly involves faculty narrators, the “study participants,” as major analysts within the project; analysts are likely to produce analyses they affirm as valid. Since faculty narrators are both “study participants” and “analysts,” they must also develop systemized reflexivity, a situation that intertwines construct and face validity for this study.

\subsubsection{Catalyic validity: transforming reality}

The fourth and final guideline is catalytic validity, or the extent to which the project “reorients, focuses, and energizes participants toward knowing reality in order to transform it” (p. 67). Although the study itself is not primarily a change initiative, the methods it employs appear to have the possibility of creating at least a change in self-understanding. For example, the second and third public transcripts from Olin faculty all discuss various aspects of their identity: Jon’s insistence on a design-centric introduction, Rob’s portrayal of himself as the “humanities” outsider at an engineering college, and Lynn’s love for precision of words have all been made visible to the narrators during the process of grounded indigenous coding. 

However, it is unclear whether this or other insights will lead to a transformation of teaching practice at Olin. Similarly, the first public transcripts from Berea faculty have started to uncover their different definitions for the word “design” and how those differences can lead to communication glitches, but it is unclear whether and how this knowledge might develop and transform the shared language of the faculty involved.

However, the “realities” of Olin and Berea’s campuses are not the ultimate focus of this study’s “impact.” My research question about understanding faculty-as-learners addresses an audience of engineering and technology faculty, faculty development professionals, and adult learning researchers. If a radically transparent and participatory perspective on research allow us to count an engaged audience member as a “participant” of sorts, any interaction with study materials and participants (including myself) has the possibility of helping them transform reality by “reorienting, focusing, or energizing” them to understand, and therefore act upon, the reality of faculty-as-learners in a different way.

\subsubsection{Reflective protocols}

In a later book, Lather presents a list of reflective questions she uses during her studies (1991, p. 84). This list includes items such as “Did I encourage ambivalence, ambiguity, and multiplicity, or did I impose order and structure?”, “Have I confronted my own evasions and raised doubts about any illusions of closure?", and “What is this fierce interest in proving the relevance of intellectual work?” Lather’s list can be seen as a sort of tool for construct validity. 

I am regularly employing a mixture of these various reflective protocols as semi-structured solo “debriefings” for myself immediately after each narrative session. The prompts I write to are a remix of Lather’s questions with Pawley’s dissertation interview summary form (2007), a set of reflective questions to be answered by the “interviewer” immediately after each “interview,” and the Situation, Affect, Interpretation, and Decision reflection framework as described by Hogan (1995). 

\begin{enumerate}
\item What were the main process issues in the narrative session? (This usually refers to scheduling, transcription, \item equipment, etc. considerations.)
\item Describe your affective experience of the narrative session.
\item How did ambivalence, ambiguity, multiplicity, order, and structure show up during the narrative session?
\item Summarize the information you got during the narrative session. What information was explicitly stated by the narrator, and what was inferred by you?
\item Anything else interesting, salient, illuminating, or important in this narrative session?
\item How does this narrative session factor into your future decision-making? In particular, what (if any) new considerations do you have for future narrative sessions, either with this narrator or another?
\end{enumerate}

These prompts are tools which regularly “push [me] toward becoming vigorously self-aware,” which Lather calls “our best shot [at validity] at present” (Lather, 1986, p. 66). The prompts themselves shift slightly over time as I find ways to improve them.

\subsection{Positionality}

With a study design so steeped in ideas like agential cuts and intersubjectivity, it is impossible for me to see my own subjectivity as anything other than unavoidably entangled in the mix. This subsection describes how I try to become conscious of, articulate, and prepare for various aspects of my subjectivity to impact my research.

\subsubsection{Triangulating within and against existing relationships to address researcher privilege}

First, as noted in the methods section, I have existing relationships with all 6 faculty narrators either as a former student (at Olin) or the frequently-visiting student of a colleague (at Berea) and want to portray them in a positive light. This is a privilege which creates shared context and trust, which in turn may facilitate taking public narratives into difficult or potentially emotional territory. However, my shared context is much higher at Olin (which I attended as part of the 2nd graduating class) than at Berea (which I have never visited for longer than 2 weeks), creating a marked difference in my “insider” role within those communities. 

Any level of shared context also means the narrators and I may fall into “insider-speak” that fails to unpack this context to others, and that I may unconsciously omit or modify interpretations in order to fit my existing impressions of the narrators and their schools. These things may occur in both analysis and data generation/collection, since the narrative choices faculty make will be dependent upon the attitude and shared context they assume I hold as a listener.

I can address the effects of existing relationships in several ways. Since the narrators and I read the narrative prompt from the start of the live transcript document during a narrative session, I can write reminders (for instance, of our goal of public transcript release) at the start of that document. Remembering our external audience may help de-emphasize me as a primary (or sole) audience for the narrative. I can also explain narrative sessions to narrators by asking them what they would say to a roomful of academic colleagues in a conference setting, which creates an alternate and somewhat more general listening target. 

I can bring in an external audience and model the behaviors of explaining our shared context to them and/or triangulating my analysis with theirs. For example, expanding campus acronyms to our CART transcriber during a narrative session is an explanation of our shared context, and asking Olin/Berea alumni or other faculty/researchers to look over data and analysis is an act of triangulation. The study design already specifically requires some degree of triangulation by having faculty narrators respond to and co-analyze each other’s data both within and across institutions.

Finally, I can be clear about my Olin/Berea positionality differences when seeking triangulation; I need to seek face validity while triangulating Berea narratives because I am a community outsider there, whereas I need to seek construct validity while triangulating Olin narratives because I may be blind to my within-community assumptions.

\subsubsection{Decentering the “transformative intellectual” of researcher privilege}

While the study blurs conventional roles, seeing all participants as fellow researchers on the project and providing the same level of access to all the data and analysis I draw upon for my dissertation, and while other people could theoretically (and legally) draw on the same dataset to write a dissertation of their own, the fact remains that I am doing this project as my dissertation. I do not assert ownership over the public dataset, which I consider to be a commons. I enthusiastically encourage others to work within that commons to repurpose and take from the commons what they need. However, I am responsible for making sure the commons contains what I need in order to complete my dissertation, and am therefore investing a high degree of time and effort into that commons that currently outstrips the quantity of contributions from any other participant.

Being able to spend this amount of time on the project is a blessing, but also tends to give me a place of privilege within the “dominance relationships in the academy of academics as the central ‘servants’ by which emancipation/truth/etc happens.” As a poststructuralist who seeks to “decenter” this image of the “transformative intellectual” (Lather, 1991, p. 47) without replacing it with another center, I use the triangulation practices described above to (imperfectly) mitigate, reflect upon, and transform my privileged position into another element that weaves into the study’s intersubjectivity. Whenever possible, I work on this study with other people instead of analyzing or writing alone; that way, while my voice may remain the most frequently present voice in the project, it rarely stands alone.

Writing is a particularly privileged space because it is an act of assigning signifiers and signifieds. In other words, no matter how many disclaimers about temporariness and non-canonicity accompany it, writing is an act of assigning meaning. Like Lather, “as I write, I face the inescapability of reductionism” (Lather, 1991, xix). Also like Lather, “my hope is to create a text open enough, evocative enough on multiple levels, that it will work in ways I cannot even anticipate” (p. xx) and constantly ask “how I can communicate my always-in-process ideas and practices in order to expand a sense of the possibilities of oppositional cultural work,” especially “given the ways the text works against itself” (p. 20).

Most of my proposed strategies for working with researcher privilege in text also play with the boundaries of what is considered “scholarly” writing. I am not yet sure where I want to go for my final dissertation document, and expect I will not know until the document is finished. I can refuse to give a definitive final analysis, and point out that the multiple analyses I submit do not condense into a single overarching “truth.” I can play with visual representations such as graphic novels, which have “signifieds” far less tightly bound to their “signs.” I can allow participant voices to speak verbatim whenever possible, as constructivist and critical researchers sometimes do. I can go further and try to write the entire document as an intensely intertextual dialogue that consists primarily of remixed and cited quotes from other works, a sort of extended quote collage about and on engineering and technology faculty in the style of T.S. Eliot’s poem “The Waste Land” (2001); such a document would require extensive annotation to be navigable.

The most likely outcome is that I will employ a combination of all of the above and more. My current plan involves generating rhizomatic fragments with and within the public dataset to serve as small prototypes for various approaches as the study progresses, and sharing, triangulating, discussing, and remixing those fragments with others on my blog and elsewhere.

\subsubsection{A category-proliferation approach to demographics and power dynamics}

Past relationship history, researcher privilege, and current demographics influence the power dynamics within this study. The boss/subordinate and mentor/mentee power dynamic of the faculty/student relationship is perhaps the strongest of these. As a graduate student, I am a formal apprentice in the academic community of practice of which they are full members, and my assigned role of “junior colleague” role has already been obvious during pilot interactions. For instance, every single narrator has spontaneously given me off-the-record advice about the PhD process at some point. In fact, since all Berea narrators received their PhDs in some variant of technology education, I have gotten more graduate school advice from them than from Olin narrators, who earned their PhD directly in the discipline they teach (history as opposed to history education, materials science as opposed to materials science education, etc). All Olin narrators have assigned me grades at some point, and know me originally as an enthusiastic but exhausted teenager who came to their offices to puzzle over my choice of major, my project team dynamic failures, my budding interest in engineering education, and my decision not to drop out of college (a decision in which all three Olin narrators were highly influential).

The main thing I can do to address power dynamics is to be aware of them and share that awareness of others. In addition to the faculty/student power dynamic, other potentially influencing demographic details include (note that, for the sake of relative conciseness, these demographic details are stated in an oversimplified categorization that goes against the poststructural tradition of proliferating categories):

\begin{enumerate}
\list At 27, I am visibly younger than all faculty narrators regardless of how formally I dress. This visually emphasizes the mentor/mentee aspect of the faculty/student dynamic. Additionally, each school has both “older and more experienced” faculty (Alan, Gary, and Lynn) and “younger and newer to academia” faculty (Mark, Jon, and Rob), at least relative to the “more experienced” group. This leads to generational identifications between faculty narrators, who have referred to the relative experience levels of other faculty narrators in public transcripts.
\list I am visibly female; all Berea narrators and all but one Olin narrator (Lynn) are male. We share the privilege of being able to present as our assigned gender.
\list I am openly heterosexual; all faculty narrators are in long-term heterosexual partnerships. We share the privilege of being able to present as the majority sexual orientation.
\list I am visibly Asian; all faculty narrators can present as white. I have not specifically asked about racial identification, but at least one narrator (Lynn) openly identifies on her public transcript as Jewish, which can be considered by some people as membership in a minority religious/racial group.
\list I studied the relatively high-prestige field of engineering, but am conducting education research which tends to be somewhat lower in social prestige. All faculty narrators have a similar interdisciplinary combination of high-prestige (engineering, technology, science, etc.) and lower-prestige (education and other humanities fields) interests and backgrounds.
\list I grew up in a middle-class family with the expectation of attending college like my parents. Most (but not all) Olin community members (faculty, students, alumni, and staff) share this class background, although Olin has always offered every admitted student either a full or half tuition scholarship. In contrast, Berea grants full scholarships to all students and only admits students with great financial need; most Berea alumni are first-generation college attendees. One Berea narrator (Gary) is also an alumnus; the others, regardless of their class background, teach in an environment where class background is a major consideration.
\list I openly identify as deaf, although I typically present as hearing when possible. All faculty narrators are hearing and can present as fully able-bodied.
\end{enumerate}

While none of these demographic details are the primary focus of analysis in the study, they can be, and have been, brought up during narrative sessions and subsequent analysis. For instance, multiple Berea narrators explained a difference in perspective by pointing out that Mark was a significantly younger and newer faculty member than Gary and Alan.These demographic features and the power dynamics they create cannot generally be changed, but I can attend to and articulate the poststructural approach of proliferating categories when these qualities come up. For instance, I could point out that even if the Berea narrators identify Gary and Alan as part of the same “generation” of faculty, their experiences of being part of that “generation” may be different, and there are also ways to see Gary and Alan as belonging to different “generations.” Instead of assuming a single shared experience of femininity, I can introduce the idea of thousands of different ways of “being female” if the topic comes up in discussion. Similar things could be done with topics of age, race, discipline, and so forth.

\subsubsection{Deafness and data access}

The ability of a deaf researcher to do qualitative fieldwork is a valid concern, given the richness of auditory information that is lost (tone of voice, nonverbal utterances such as laughter and sighs, pauses, and so forth). CART (realtime transcription) providers can sometimes note pauses and nonverbal utterances on a transcript, but it is by no means a substitute for the full experience of being a hearing person in the conversation. Generally, I risk being — and being seen as — less able to communicate and connect with narrators than a hearing listener would be, especially since most of my data collection is done remotely over conference calls without the aid of lipreading.

I am missing information. There is no way around this. I address my deafness primarily by being upfront with narrators about it; all of them have been extremely accommodating. During narrative sessions, I have chosen to ask narrators to repeat an unclear phrase if I’ve missed it, or to wait for our CART provider to catch up or reconnect if I need to read the realtime transcript. However, these choices also interrupt narrative flow. Because of that, I have also sometimes chosen to barrel ahead with intense concentration and a high degree of guessing in the absence of a CART provider, for example when the CART provider’s connection to our Skype call or the Google Doc holding the live transcript flickers despite our best attempts to set up a reliable connection. Fortunately, these moments of lost connection (both technical and interpersonal) are rare and usually only a few minutes long at most.

Between the narrators, the CART providers, and myself, we have developed transcript reconstruction strategies in response to different situations. For example, when a scheduling error led to an entire narrative session with Alan being conducted without any live transcription, I took copious handwritten notes that Alan and I then used for grounded indigenous coding. I also audio-recorded the session and sent it in for after-the-fact professional transcription, then pieced the professional transcript together with my handwritten notes to create a sort of “live transcript equivalent” Alan could then edit into the public transcript. In general, when a transcript is reconstructed, I note the details of its reconstruction in the public transcript version. Whenever possible, I preserve and release both the pre- and post-reconstruction transcript versions.

Working with my deafness and finding transcript reconstruction workarounds has also emphasized the study’s intertextuality and the unusually improvisational, collaborative, and high-agency roles of participants (including myself) as compared to “interview subjects” and “researchers” in most qualitative studies. For example, when our CART provider Beth suddenly had a connection outage in the middle of one of Jon’s narrative sessions, Jon and I noticed almost immediately. Since I can type with almost-perfect accuracy at well over 135wpm and can approach realtime transcription of regular speech with a considerable sacrifice of that accuracy, I spontaneously took over realtime transcription duties, prompting Jon to continue with his narration. By typing what I was hearing, I also simultaneously showed Jon how I understood his speech. Jon slowed his speech rate, watched the live transcript carefully as it was generated to warn me of major errors, and re-summarized that portion of his narrative to Beth when she reconnected several minutes later. Immediately after that narrative session, I corrected typing errors for the portion of the narrative I had transcribed.

On yet another occasion, Lynn began to type and correct minor inaccuracies in her live transcript while she spoke. Halfway through that narrative session, Lynn pointed out that she and I both type nearly as fast as we speak, and we agreed to conduct the second half of the session in text chat so that she could directly control the precision of her live transcript wording. Lynn’s second session was conducted entirely in text chat, and this was so successful that we are discussing what this means for her narrative sessions going forward.

No communication or transcription process is “perfect.” Hearing conversation partners still miss material and need to ask each other to repeat statements. Transcripts are always translations of speech events; as with all translations, things are both lost and gained. It is a tradeoff between (among other things) the richness of auditory information and the convenience of far more reproducible, transmissible, and easily analyzable data artifact (Porter, 1995). Finally, as described in the earlier section on CART, finding ways to work with my deafness inadvertently led to the development grounded indigenous coding, now a major methodological component of this study’s design and one of its contributions to research practice.

Working with my deafness is not always easy and can occasionally be frustrating. However, I have personally found it to be far more fascinating and fun than frustrating, and narrators have voiced similar sentiments when I have asked. The most imporant thing is that we address the situation in a way that feels authentic to all those involved, and that we make our process visible by documenting it with notes in and on the artifacts that are shaped by these dynamics.

\section{Conclusion}

This document has described a proposed study to address the research question “how can the interacting narratives of engineering and technology faculty inform our understanding of faculty as learners?” It has laid out the rationale for addressing such a research question and described the interacting theoretical structures of cognitive apprenticeship (framework), narrative analysis (methodology), and poststructuralism (paradigm) that underlie the study design. It has walked through various aspects of how the study will be executed: who the faculty narrator participants are and why, how and why data will be collected via narrative sessions with realtime transcription and grounded indigenous coding, and how and why data will be made publicly available in a radically transparent approach to research. It has described the generation of public transcripts, a cast of character sketches, and rhizomatic fragments as forms of analysis that address the research question. Finally, this document has discussed concerns of logistics, validity, and positionality that may come up during the course of the study.

Although this document will be frozen and submitted, the thinking and the study it depicts continue to evolve. As Susanne Kappeler has said, and Patti Lather has borrowed for the ending of her own book: “I do not really wish to conclude and sum up, rounding off the argument so as to dump it in a nutshell on the reader. A lot more could be said about any of the topics I have touched upon. I have meant to ask the questions, to break out of the frame... The point is not a set of answers, but making possible a different practice” (Kappeler, 1986, p. 212; Lather, 1991, p. 159).
