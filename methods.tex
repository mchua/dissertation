\section{Methods}

In this section, we will walk through the plan for carrying out the proposed research project. We will describe the people and institutions who will participate in this study of faculty-of-learners, explore why we chose to have them tell stories about design thinking curricular revisions (DCTRs), and explain why their participation is non-anonymous. We will describe and justify the study's radically transparent approach to research. We will contextualize narrative analysis as a data collection method, then walk through the steps of carrying out a narrative elicitation session in order to collect that data. We will discuss the use of realtime transcription and the new methodological contribution of grounded indigneous coding, then explore how this method blurs boundaries between data collection and analysis. Finally, we will walk through the steps of analysis and show how creating public transcripts, tracing a cast of characters through the narratives told, and writing rhizomatic fragments will address the research question and help us understand faculty-as-learners.

\subsection{Participants}

\subsubsection{Participating institutions: Olin College and Berea College}

Since I wanted to see if narratives could inform our understanding of faculty-as-learners in a way that would transfer across institutions, I am recruiting faculty narrators from two schools: Olin College in Needham MA, where I earned my B.S. in 2007, and Berea College in Berea KY, where I have visited and collaborated with committee member Matt Jadud on multiple occasions. Existing relationships made it easier to gain access to the level of trust and commitment needed for an intensive narrative study. Both schools are small, student-centered, undergraduate teaching colleges with 4-year engineering (Olin) or technology (Berea) programs that have gone through a curriculum revision process focused on integrating “design thinking” through their entire program.

\subsubsection{Participant narratives focus on design thinking curriculum revisions (DTCRs)}

In order to gather narratives with a point of commonality for thematic coding, I am asking faculty narrators to focus on their participation in that “design thinking” curriculum revision (DTCR) process. The choice of DTCRs as a narrative focus is semi-arbitrary; it could have been another shared experience such as “mentoring senior capstone projects” or “integrating writing across the curriculum” while still addressing the research question of faculty-as-learners. In other words, the point of commonality is an artifact rather than the focus of analysis; this study focuses on understanding faculty-as-learners rather than how faculty-as-learners make sense of DTCRs.

Criteria for recruiting faculty narrators

In order to ensure all faculty narrators could contribute in a similar manner to the narrative accrual on DTCRs, I required participants to have been deeply involved in the DTCR at their respective institution and have subsequently taught courses affected by the revision. Berea College had 3 faculty members within Technology and Applied Design (TAD) who fit these criteria, and all were willing to participate, so 3 faculty members fitting the criteria from Olin College were chosen to provide a parallel. Having 3 narrators from each institution allows for a within-institution version of the intertextual technique of asking one narrator to compare and contrast narratives from two others. This results in a total of 6 participating faculty.

Additionally, although it was not a recruitment criteria, all participating faculty happen to be cross-disciplinary. Each of them has both a formal academic background in engineering or technology and either a formal credential or current research focus in engineering or technology education. The cross-disciplinary and education-related backgrounds of participants is an unforeseen advantage that allows our conversations to delve deeper into shared meaning-making than may otherwise have been possible, since “diverse perspectives can create productive conflict and enable new paradigms of thought” (Adams & Forin, p. 123).

\subsubsection{Non-anonymous participation}

Because of the small size and unique nature of their institutions, I realized early on that faculty narrators would be highly identifiable. Instead of anonymizing both participants and institutions, which would destroy the narrative particularity emphasized by Bruner (1991, p. 6-7), I made “willingness to be fully identified” part of the study participation criteria. Faculty narrators for this study are (in alphabetical order by first name):

\begin{description}
\item[“Alan”] - Alan D. Mills, Professor (and Chair) of Technology and Applied Design at Berea College
\item[“Gary”] - Gary S. Mahoney, Professor (and former Chair and alumnus) of Technology and Applied Design at Berea College
\item[“Jon”] - Jonathan Stolk, Professor of Mechanical Engineering and Materials Science at Olin College
\item[“Lynn”] - Lynn Andrea Stein, Professor of Computer and Cognitive Science (and Associate Dean and Director of the Collaboratory) at Olin College
\item[“Mark”] - Mark P. Mahoney (no relation to Gary), Assistant Professor of Technology and Applied Design at Berea College
\item[“Rob”] - Robert Martello, Professor of the History of Science and Technology at Olin College
\end{description}

\subsection{Data collection}

\subsubsection{Radical transparency and open-licensing of data}

This project takes a radically transparent approach to qualitative research inspired by open communities such as Linux and Wikipedia. Specifically, this means I will transfer copyright of transcripts to faculty narrators, who will then edit and release them under a Creative Commons Attribution Share-Alike license (Creative Commons, 2013) that enables sharing and reuse under the conditions that original work be fully attributed and any remixes be released under a simiarly open license. All analysis will be done on this open dataset and likewise shown in public. By publishing raw data, intermediate analyses, and final study outputs under an open license, the project "source code" becomes not only readable but editable and reusable, enabling legitimate peripheral participation (Lave & Wenger, 1991).

My radically transparent approach takes advantage of widespread online communications, which have caused researchers to rethink what it means to transmit scholarly impact (Priem, 2013). A poststructuralist might say that internet technologies "explode the space-time limits of messages… despatialize certain kinds of work, enable signifiers to float in relation to [signifieds], become a substitute for certain forms of social relations, provide a new relation between author and text” (Poster, 1987-1988, p. 121). They do so by removing many of the format and throughput limitations historically imposed by print. 

Porter (1995) depicts the use of print journals as a long-distance communications technology meant to bridge local communities of research into a global society of knowledge. In turn, the format and throughput limitations of print required an adherence to strict scripts, formulae, and computations. The desire to scale impact transforms research into a "highly disciplined discourse" that can "produce knowledge independent of the particular people who make it" (p. ix), emphasizing replicable mechanical objectivity over personal trust relationships and silencing a multiplicity of voices for the sake of scale.

Ironically, these practices meant to spread the impact of research may have also decreased research's ability to impact practice, since it is precisely those personal trust relationships that lead practitioners to adopt new ideas (Borrego, Froyd, & Hall, 2010; Fincher, Richards, Finlay, Sharp, & Falconer, 2012). The use of replication instructions meant to make quality research "open" to "anyone" may have also erected access barriers to both understanding and conducting research, since instructions are only useful to those who have the background, time, and equipment to work through them, The emphasis on methods and instruments that can be replicated over time and distance (Porter, 1995, p. 226-227) also leads to a privileging of quantitative over qualitative studies, since numbers can be transmitted and checked far more easily than hundreds of pages of fieldnotes on a non-replicable historical event.

By taking a radically transparent approach to qualitative research, this project contributes to the dialogue on rethinking scholarly impact. It decenters what Lather calls the "transformative intellectual" and "[interrupts] the dominance relationships in the academy of academics as the central "servants" by which emancipation/truth/etc happens" (1991, p. 47). With the academic researcher decentered, non-traditional audiences are invited to join the questioning of research practices and their effect on access; to participate in this questioning is also to practice in research itself.

This is not to say that my study is fully accessible. Internet access, English fluency, and leisure time are still required to interact with the project. As the only person present for all narrative sessions, I experience the pauses, blinks, and tone of voice which are lost in the choice to provide only text transcripts and not audio or video recordings. I will address this gap of researcher privilege by limiting my analysis to the "open" data of the project and acknowledging the limitations and fragmentation of what I do expose (Porter, 1995).

\subsubsection{Overview of narrative analysis as a method}

In the literature review earlier in this document, we explored the underlying philosophical perspectives of narrative analysis as a methodology. Here, we examine narrative analysis as a concrete method for carrying out research that is shaped by the underlying philosophical perspectives of the methodology.

Narrative analysis applies techniques from literary studies to the stories people tell in order to understand how people construct meaning from the world. Bruner (1991) describes narratives as things through which people construct social systems by expressing and extracting meaning, but which cannot be interpreted as sequences of strict causality because of the free will their characters possess. Instead, readers interpret the reasoning behind a story according to the context surrounding them, a context they may or may not be aware of. The communal interpretation of a narrative or accrual of narratives is thus a way to share and negotiate our differences in perspective without necessarily collapsing those differences. This is an important feature for this research project, which seeks to open and deepen dialogue across different institutional contexts on a topic whose interpretation is often tacit and personal and therefore difficult to transfer.

When our shared narratives are about our own actions, we end up engaging in the reflection-in-action (Schön, 1983) that cognitive apprenticeship literature so highly values (Collins, Brown, & Holum, 1991). Reflection-in-action is a place of generative tension; while standard and shared meanings are indeed reified by community narrative collections, Mishler (1986) also describes this participation in community discourse as helping us avoid the trap of assuming those standard meanings out of habit, and instead allowing meaning to emerge through and be realized in the discourse itself (p. 65). The aim is not to search “for the best or most authentic answer,” but rather to “systematically activate applicable ways of knowing – the possible answers... as diverse and contradictory as they might be” (Holstein & Gubrium, 1995, p. 37). “Our individual autobiographies... depend on being placed within a continuity provided by a constructed and shared social history in which we locate our Selves and individual continuities. It is a sense of belonging to this canonical past that permits us to form our own narratives of deviation while maintaining complicity with the canon" (Bruner, 1991, p. 20). This research project seeks to elicit such reflexive autobiographies on teaching and to help their narrators collaboratively place them into a shared social history that will be accessible to the public.

\subsubsection{Data collection model: narrative elicitation sessions}

This project’s specific implementation of narrative analysis as method uses repeated storytelling as a data collection model. Since the project employs narrative method(ology) from within a poststructuralist paradigm, I deliberately chose not to use the term “unstructured interviews” for each dialogical narrative episode in order to deliberately step away from the researcher/subject positionality and power dynamic typically associated with the word “interview.” Instead, I call them narrative elicitation sessions, sometimes shortened to simply narrative sessions.

Each faculty participant is asked to engage in a series of narrative elicitation sessions, each beginning with a prompt. For a narrator’s first session, the prompts are “How would you like to introduce yourself to the readers of this study?” and “Tell me the story of design thinking curriculum revision (DTCR) at your school and how you came to be involved in that” (or similar wordings). This results in a narrative, which is then edited into the prompt for the second session. In general, prompts beyond the first session are remixes of excerpts from the previous sessions of some subset of the narrators, and the current narrator is asked to respond in any way they wish. This results in an intertextual narrative.

For example, Lynn was asked in her first session how she would tell the story of DTCR at Olin. Part of her response was:

LYNN: [If I were to write a book about the curriculum revision,] Chapter 1 [would be called] "We need a design experience in the fourth semester."

For his first session, Rob was also asked how he would tell the story of DTCR at Olin. However, for his second session, his prompt contained sections of Lynn’s first session transcript including the section above. He responded:

ROB: Chapter 1, she is saying there is some need for design experience. That is fascinating for me to hear. I wasn’t aware of any of the details of that discussion.

Subsequently, Jon entered the study, and his first session prompt was also to tell the story of DTCR at Olin. However, his second session prompt consisted of both Lynn’s transcript and Rob’s reply to Lynn’s transcript. Jon responded:

JON: So what Lynn said is exactly what I remember. I think I described this in my last interview or maybe the first one... It’s interesting to see Rob’s comments too that he wasn’t aware of the detail...

As the above example begins to show, using these prompts for the narrative elicitation sessions leads to faculty narrators co-analyzing each other's experiences. The resulting emerging narrative accrual consists of interlinked narratives that overlap, intertwine, and make-meaning-of each other.

\subsubsection{Participant commitment: 6 narrative sessions and 1 debrief}

Although faculty members write intertextually all the time (when quoting references in scholarly papers) and have frequent intersubjective interactions (co-analyzing data on a research project), it may be disorienting for them to become conscious of these behaviors and try to deliberately employ them in a narrative elicitation session. We can optionally scaffold this process by choosing which subsets of the existing narrative accrual to draw from when creating prompts.

For this study, I ask faculty narrators to participate in 6 narrative elicitation sessions that move them from (1) responding to themselves to (2) responding to familiar colleagues from their institution to (3) responding to unfamiliar colleagues from a different institution, and then back through the same progression so that they end by reflecting on their own narratives once again. The series of sessions and their prompts are as follows:

Introduction to the study; solo DTCR narrative starting from a blank slate
Respond to yourself (prompt source: narrative #1 from same narrator)
Respond to colleagues at your institution (prompt source: narratives #1-3 from same-institution narrators)
Respond to faculty at another institution (prompt source: narratives #1-4 from different-institution narrators)
Respond to colleagues at your institution (prompt source: narratives #1-5 from same-institution narrators)
Respond to yourself (prompt source: narratives #1-5 from same narrator)

Narrative elicitation sessions take place over the course of approximately one year to allow adequate time for scheduling. Each session is scheduled to last 90 minutes. The only exception to this is the first session, which has an additional 30 minutes at the start in order to give time to explain the study, answer questions, and do IRB paperwork. This results in approximately 54 hours of data collection time (6 narrators x 6 sessions x 90 minutes). The actual narrative data recorded will be somewhat less, allowing for people arriving late, ending early, taking breaks, etc.

I have also asked faculty narrators to participate in separate individual debriefs after they complete all 6 narrative elicitation sessions. These conversations will focus on what it was like to participate in the study and any suggestions they have for further work or methodological improvement. They will probably last 30-60 minutes, and the transcripts from these conversations will not be analyzed for my dissertation; they are primarily to provide closure.

\subsubsection{Narrative elicitation session setup: realtime transcription with CART}

Qualitative data is typically audio-recorded and transcribed after the fact by a listening researcher. However, I am deaf. While I speak fluently, lipread well enough to comfortably conduct in-person interviews, and can sometimes conduct phone interviews on familiar topics if background noise is minimal, this is exhausting, inaccurate, and not an ideal setup for an extended series of interactive narrative sessions. If I wanted to do qualitative research, I needed to find a way to collect and transcribe high-quality data without exhausting myself.

At the time, I had just started using CART (Communication Access Realtime Translation) for accessibility in my graduate classes. CART consists of a trained provider with a stenographic keyboard listening to the discussion and transcribing it in realtime onto a display for clients to view. Far more accurate than modern-day speech recognition software, CART providers can handle accents, technical terminology, homophones, laughter and other non-word noises, speaker changes, and other auditorily difficult situations — such as my data collection, which would feature academics speaking rapidly and using technological, pedagogical, and psychological terms.

I realized that if I used CART for my narrative elicitation sessions, I would not only have a far easier time following the conversations, I would also have no need to transcribe them afterwards. CART providers can either listen to the discussion in-person, or remotely via a conference call. Either way, the effect for the client is that of having subtitles for real life. I chose remote CART for convenience of scheduling and location. At $60-120 per hour, the cost was comparable to outsourcing transcription. The process can be thought of as similar to hiring an interpreter for foreign-language interviews; a CART provider is essentially a speech-to-text interpreter. The setup for narrative elicitation sessions became as follows:

I arrange a session time with the narrator and the CART provider at least a week in advance and send the CART provider a list of anticipated vocabulary (acronyms, proper names, specialized terms, etc.) to aid with understanding.
At the appointed time, the narrator and I connect, either by arriving at the session location (if I was conducting the interview in-person) or begin our call (if I was conducting the interview remotely).
I open the session prompt in a collaborative editor (Google Docs) on my laptop so that both the narrator and I can read it. If the interview is remote, the narrator opens the document on their own computer so that we can both read the document while speaking.
I connect our pre-arranged CART provider to both the call and the document. The CART provider begins to transcribe our discussion conversation into the document, allowing both the narrator and myself to read and edit the transcript as it was generated (for instance, if a name was spelled incorrectly or a term was misheard).

With these transcription and accessibility details taken care of, the narrative elicitation session proceeds as previously described. However, realtime transcription also inadvertently enables what may become the main methodological contribution of this study: grounded indigenous coding.

\subsubsection{Grounded indigenous coding within the narrative elicitation}

At any point during a narrative elicitation session, the use of CART makes the full transcript of the session up to that point available to both the narrator and myself for commenting and editing. This enables us to engage in grounded indigenous coding, a new contribution to qualitative methodology that leads to opportunities for rich narrative intertextuality. Grounded indigenous coding combines the precision of member-checking grounded in verbatim transcripts with the immediacy of analyzing data while generating it, which Holstein and Gubrium call indigenous coding (1995). This dissertation is, to my knowledge, the first study to deliberately utilize grounded indigenous coding and the first study to give it a name.

Since grounded indigenous coding is a new concept, we will walk through a concrete example before unpacking it with prior literature and examining its significance as a methodological contribution. The following example is drawn from the start of Jon’s first narrative session, where he introduces himself:

JON: My name is Jon Stolk. I have been at Olin since 2001... I got to Olin a little bit before students arrived.

Approximately 40 minutes later, during the same narrative session, Jon reads the above line in his own transcript and begins a self-analysis with the following comment:

JON: It is important for me to communicate to people that I arrived before students arrived at Olin. I notice that I always do this. I always have to get that into my introduction.

Jon is clarifying and reflecting upon what he’s saying while he’s saying it, which Holstein and Gubrium classify as a form of data analysis called indigenous coding. Phrases like “you mean...” or “to sum it up...” or “one way of thinking about it…” are typical signals of indigenous coding (Holstein & Gubrium, 1995, p. 56). Narrators see new patterns while reflecting on the stories they have just told, occasionally eliciting further thoughts.

Although these speech acts take place naturally during conversation, I also deliberately choose to make them visible to faculty narrators whenever possible, subverting the usual researcher/subject power dynamic. By naming these speech acts as moments of co-analysis, indigenous coding transforms narrators from “subjects” to “fellow researchers.” Since all the narrators for this study run their own research projects as a non-trivial part of their faculty jobs, indigenous coding also moves the narrative session into a familiar setting within their locus of control.

Jon’s self-analysis is also grounded in his verbatim transcript, in contrast to the memory-based nature of most indigenous coding. This allows the narrative session to function as a preliminary member check of transcript accuracy, a technique used in some qualitative studies to confirm transcript accuracy and involve subjects more deeply in the co-creation of knowledge. Having the transcript as a concrete boundary object allows precise, detailed analysis of phrasings and re-phrasings, specific vocabulary, and things the narrator may not realize they’ve said. However, transcription turnaround times typically require member checks be scheduled as separate sessions, trading immediacy and scheduling convenience for a grounded analysis.

By reducing transcript turnaround time to 5-10 seconds (the average remote CART delay), we remove the tradeoff between immediacy and precision. By making transcription an important study design element, we uncover it as a component of research methodology whose effects on data often go ignored and unthought-of (Lapadat &Lindsay, 1999). By not temporally and spatially separating “data collection” from “data analysis,” we also reveal the arbitrary nature of the slice researchers often make to separate the two.

\subsection{Analysis}

As we have already seen, the intersubjectivity of this study blurs the distinction between “data collection” and “data analysis.” Any boundaries drawn between “collection” and “analysis” or between different “stages” or “types” of analysis are agential cuts that enact agential separability, researcher-constructed distinctions between an analyzed phenomena and its external observer. Karen Barad used agential separability to make-visible the performativity of poststructuralism, a paradigm that “insists on understanding thinking, observing, and theorizing as practices of engagement with, and as part of, the world in which we have our being… We are not outside observers of the world. Neither are we simply located at particular places in the world; rather, we are part of the world in its ongoing intra-activity" (2007, p. 133, 184). However, presenting ourselves as outside observers and analyzers of our data is necessary to “secure the possibility of reproducibility and unambiguous communication" (p. 174) and allows us to transmit ideas across a larger society of individuals we do not personally know (Porter, 1995).

Because of this necessity, I have subdivided the analysis plan into several stages of data transformation. Each stage addresses an aspect of my research question, which focuses on understanding faculty as learners.

\subsubsection{Narrative elicitation session: produces public transcript, highlights agency}

As explained in the data collection section of this document, a narrative elicitation session produces a live transcript of the faculty narrator engaging in both narration and grounded indigenous coding. This transcript is raw and unedited, and may contain typos or other material the faculty narrator wishes to modify or delete before releasing it as public data. They are therefore given time within the narrative session to review the transcript (resulting in more grounded indigenous coding) as well as time after the session (typically 1 week, but more if needed) to review and edit their transcript until they feel comfortable releasing it publicly.

The public transcript is the narrator-approved version that is released online under an open license. Public transcripts are fully identified (“These words are from Dr. Alan Mills of Berea College”) and are usually very close to, if not identical with, the live transcript. Once the faculty narrator is finished editing, they enter their public transcript into this study's public dataset. Faculty narrators may also identify portions of their transcript that they want to save as study data but do not wish to be identified with. In these cases, that part of the live transcript goes in the private dataset of a later study for de-identified analysis. The remainder of this discussion on analysis will deal only with the public dataset, as it is the only material analyzed for this study.

Edits and deletions from live transcripts never enter any dataset, public or private. However, conversations that occur while editing the live transcript into the public transcript are also part of the narrative elicitation session, and are therefore transcribed in realtime, becoming part of the live transcript as well. Depending on how faculty narrators choose to enact their agential cut, their public transcript may mention the existence of an edit (“Let’s take out the name here and just say ‘a student’”), or the conversation about editing itself might be removed entirely.

The creation of the public transcript can be regarded as an analytical act that highlights a faculty narrator’s agency in determining how to present themselves. It obeys Michael Apple’s exhortation to shift the researcher’s role “from being [a] universalizing spokesperson to acting as cultural workers whose task is to take away the barriers that prevent people from speaking for themselves” (Lather, 1991, p. ix) In terms of our research question about understanding faculty as learners , the public transcript shows us the data our narrators would like us to use in order to understand them, providing a starting point for the rest of our analysis of them as learners.

\subsubsection{Tracing voice and characters: produces a cast list, highlights narrative representations of community performance}

Once the public transcript has been released, we can trace the voice in which the narrative is being told and who it is being told about, two separate analytical procedures drawn from a 2008 Doucet and Mauthner study. Tracing voice consists of attending to the narrator and “how this person speaks about her/himself... [we use] a coloured pencil to trace the ‘I’ in the interview transcripts. This process centres our attention on the active ‘I’ who is telling the story... highlighting where the respondent might be emotionally or intellectually struggling to say something. It also identifies those places where the respondent shifts between ‘I’, ‘we’, ‘you’ or ‘it’, which can signal varied meanings in the respondent’s perceptions of self” (p. 405-406).
For example, in the fictionalized public transcript excerpt given below, we hear the narrator shift from a present-tense singular pronoun (”I don’t know”) to externally describing a hypothetical alternate-universe self (”if I would have necessarily gone…”) to speaking with the voice of that hypothetical alternate-universe self (”This is what I will do this week.”) to the sudden use of a plural pronoun (”we didn’t really have the resources.”)

NARRATOR: I don't know if there was a big book of like, “this is how to teach the class.” If I would have necessarily gone ah, you know, here is the recipe. This is what I will do this week. This is what I will do next week to kind of see the big story so that I could confidently do it... we didn't really have the resources.

Tracing other characters is a “reading for social networks... and close and intimate relations” (p. 406). It can be thought of as the answer to the following question: If this were a theatre monologue from a full play, what would the “cast of characters” on the playbill read? Using another fictionalized public transcript excerpt for illustration, we might see one character as an unnamed “Prior Professor” who taught the class last year then suddenly and unexpectedly was unable to teach it at the start of term.

NARRATOR: So it's been an interesting story in terms of not having a fair amount of information about what was done in the past, and not knowing why he chose the things that he chose to read, and not really seeing the pattern and not really even seeing a, you know, a set of questions that he might have had in his mind. It was just wide open. You know, the downside of that is I didn't really feel comfortable just saying I will do exactly what he did because I don't know exactly what he did. [Laughter] ...then the old TAs (teaching assistants) started getting e-mails [from me] about can you tell me how he used time in class...

Even with just this excerpt, we can begin to write a character sketch of Prior Professor. He has left a reading list (“the things he chose to read”) behind, but no explanation for the logic behind their design (“not knowing why he chose the things that he chose”). Since the narrator is emailing Prior Professor’s teaching assistants (“the old TAs started getting e-mails about can you tell me how he used time in class”), we can hypothesize Prior Professor can no longer be contacted

After the character sketch of Prior Professor is written, the TAs could be taken as another character (or group of characters) in the story, with a character sketch of their own. Character sketches intertwine as they are written; the TAs used to work for Prior Professor, the narrator is taking over a teaching responsibility from Prior Professor, the narrator is trying to contact the TAs, and so forth.
These two analytical procedures help us understand faculty-as-learners in several different ways. First of all, it portrays them as narrators who use the semiotic conventions of their culture to put together language that simultaneously assembles them (Parker, 2004, p. 90) as they weave their personal meaning-making into the narrative accrual. Secondly, it portrays them as interacting with other characters in a narrative that takes place within their community of practice, allowing us to see their stories as folk tales of cognitive apprenticeship and situated learning. Finally, it helps us understand their narrative construction of reality as intersubjective; since faculty are iterating between telling and reading each other’s narratives of shared experiences, they appear frequently as characters within each other’s stories and reshape those characterizations as they go. “Wait, that’s how she sees me? Wait, that’s how he sees her? Let me rethink the way I see myself and them again.”

\subsubsection{Linking discourses: produces rhizomatic fragments, highlights pluralities of ways to understand faculty-as-learners}

Once the public transcript has been released, we can take an inductive approach to the data, rummaging our grubby hands through the mess and playing with the connections that pop out. This is an intertextual act of “linking into discourses,” which differs from the typical inductive approach of “sorting into themes” (Parker, 2004, p. 100-101). “Emergent themes” are a common “building block” for inductive analysis (Williams, 2008) and are often depicted as being passively “allowed” to emerge before being categorized.

In contrast, poststructural texts are full of “pastiche, montage, collage, bricolage, and the deliberate conglomerizing of purposes,” all highly active events that “fight the tendency for our categories to congeal” (Lather, 1991, p.10, 13). Instead of sorting emergent ideas into as few buckets as possible, poststructuralists proliferate categories until there too many to fit neatly into an overarching metanarrative (Lyotard, 1984). Intertextuality and the self-referential nature of sign systems such as language mean that the act of linking into discourses does not have a set beginning or end; even a text with zero citations in APA format draws from ideas and vocabularies that existed before it.

Linking discourses endlessly weaves and re-weaves our data into and within the massive fabric of the world’s information, which leaves us with an overwhelming Library of Data We Haven’t Analyzed Yet, to adapt a phrase from Italo Calvino’s If On A Winter’s Night A Traveler (1982). Just as we enacted agential cuts around our data so that a set of data would exist for us to analyze, we must enact agential cuts within our data to snip out finite artifacts that are small enough to analyze. Inspired by the Capitalism and Schizophrenia project’s description and example of poststructuralist theory and research as a nonhierarchical rhizome that can be entered, navigated, represented, and exited from many points, in many ways, and with many interpretations (Deleuze & Guattari, 1987; Deleuze & Guattari, 2009), I call these little snippets rhizomatic fragments.

We can create an example rhizomatic fragment with the intertextual transcript excerpts used earlier to introduce narrative elicitation sessions as a data collection model.

Lynn said that Chapter 1 [of a book about Olin's curriculum revision would be called] "We need a design experience in the fourth semester." That is exactly what Jon remembers. Rob wasn't aware of any of the details, so Jon thinks it’s interesting to see Rob's comments.

We can create more than one rhizomatic fragment with the same tiny set of quotes. Note the difference in formats between example fragments, which could also be graphical, written in multiple languages, structured as sonnets, etc.

LYNN: We need a design experience in the fourth semester.
ROB: She is saying there is some need for design experience.
JON: So what Lynn said is exactly what I remember. [Rob]...
ROB AND JON: ...wasn't aware of the details.
ROB: That is fascinating for me to hear.
JON: It's interesting to see Rob's comments too.

Rhizomatic fragments engage us in deconstruction, “[encouraging] a multiplicity of readings by demonstrating how we cannot exhaust the meaning of the text, how a text can participate in multiple meanings without being reduced to any one, and how our different positionalities affect our reading of it" (Lather, 1991, p. 145). The rhizomatic fragment examples above drew from a tiny set of just three speaker utterances. Access to the full public dataset of up to 36 narrative session transcripts, each with hundreds of speaker utterances, opens up the possibility of many more. In fact, all of this study’s narrative session prompts are rhizomatic fragments; as they are remixes of excerpts from the previous sessions of some subset of the narrators.

Rhizomatic fragments may include or even center around theoretical constructs that could be called “themes.” One or more rhizomatic fragments from this study might link into the discourse of cognitive apprenticeship. One or more might link into the discourse around narrative, perhaps using some discourses within engineering education and some completely outside of it. Rhizomatic fragments from this study could also link into discourses not fully fleshed-out in this document. For instance, Baxter-Magolda and King’s ideas on self-authorship were briefly mentioned in several earlier sections, but not unpacked; a rhizomatic fragment would be one place to explore further connections with that set of literature. I discussed modeling as a think-aloud protocol for sharing expertise in a cognitive apprenticeship; a rhizomatic fragment could further delve into literature on the difficulties of sharing tacit knowledge. Another rhizomatic fragment could explore literature on various conceptions of design thinking, since design thinking curricular revisions are a point of commonality for all 36 narrative sessions in this study.

Authoring these fragments, and explicitly employing radically transparent practices so others can do the same, helps address the research question of understanding faculty-as-learners by proliferating a host of ways to see them as learners. A poststructural narrative accrual composed of rhizomatic fragments does not need to converge upon a single version of “truth.” On the contrary, it “[creates] a multiply voiced text that accumulate[s] meaning as the text proceed[s] in a way that goes beyond the pages of the book” (Lather, 2008, p. 2). Authoring rhizomatic fragments can be seen, via an agential cut, as a form of analysis.Far from being a “failure of interpretive responsibility” to “analyze… what [the narrators’ words] really meant” (p. 2), the rhizomatic fragments produced by linking discourses highlight the pluralities of ways we can constantly make and re-make our understanding.
