\section{Literature review}

This section presents a more in-depth view of the three main bodies of literature used to frame the study: cognitive apprenticeship, narrative analysis, and poststructuralism. Each provides a different way of looking at faculty-as-learners. Cognitive apprenticeship is a framework that helps us see faculty as learners within a community. Narrative analysis is a methodology that allows us to draw on their experiences as self-directed adult learners. Both cognitive apprenticeship and narrative analysis are already being used within engineering education research. Finally, we introduce the poststructuralist paradigm as a way of integrating the framework of cognitive apprenticeship and the methodology of narrative analysis. Taken together, these bodies of literature help us frame faculty-as-learners in a way that articulates and values the liminal experiences they engage in to transform engineering and technology education.

\subsection{Viewing faculty-as-learners as participants in a cognitive apprenticeship}

One way of thinking about faculty as learners is to view them as participants in a cognitive apprenticeship (Collins, Brown, & Newman, 1987). The cognitive apprenticeship framework was developed alongside theories of situated cognition and communities of practice (CoP) in the late 1980s and early 1990s. If situated learning says that all knowledge is contextual (Brown et al, 1989), and communities of practice are the domain-specific groups of practitioners with whom we share a fellowship (Wenger, 1999), then cognitive apprenticeships are how newcomers learn a contextually-situated intellectual craft by working with and observing others in a CoP. Faculty, especially novice faculty or those new to a particular school, can be thought of as apprentices who learn from watching more-seasoned faculty navigate the practice of teaching engineering.

\subsubsection{Existing use of the cognitive apprenticeship in engineering education research}

Situated learning and CoPs are not new ideas in engineering education. In “Situated Engineering Learning: Bridging Engineering Education Research and the Learning Sciences,” Johri and Olds (2011) provide numerous examples of how situated learning is already embedded in engineering education by virtue of its emphasis on tangible, real-world, hands-on project work. CoP theory is even more widely used in engineering education, including usage specifically geared towards faculty development. For example, the NSF-funded Rigorous Research in Engineering Education (RREE) workshops were based on a CoP model (Streveler, Smith, & Miller, 2005). Engineering education conferences feature workshops and special sessions with titles such as “Feminist engineering education: building a community of practice” (Pawley et al., 2009) and “Communities in practice in engineering education: what are we learning?” (Adams et al., 2005). While not limited to engineering or technology faculty, the Faculty Learning Community (FLC) movement within faculty development has been described as a specific type of CoP (Cox, 2004). Finally, the American Society of Engineering Education (ASEE) is developing a NSF-funded virtual CoP model for faculty, citing familiar-sounding frustrations with the “inherent limitations” of the “develop-disseminate” model in which researchers develop new materials “and then try to convince others to use them… without any follow-up activity,” (Pimmel et al., 2013, p. 2).

/subsubsection{Cognitive apprenticeships as communal and situated learning within a community of practice}

Apprenticeship-style learning takes place within the practice it seeks to modify. The creators of cognitive apprenticeship theory were education researchers inspired by anthropological observations of traditional craft apprenticeships such as building furniture or delivering babies. They noticed that apprentices developed their skills in a meaningful context where their novice efforts clearly contributed to the “real practice” of their craft and the building of their skill in that craft (Brown, Collins, & Duguid, 1989). Reading or talking about designing engineering curricula is not the same thing as actually designing or teaching it, one possible reason the engineering education research-to-practice transfer problem is more difficult than a information dissemination perspective might make it sound (Siddiqui & Adams, 2013).

Cognitive apprenticeships also differ from the “information dissemination” approach through their emphasis on the social aspect of learning; learners are surrounded by other learners and multiple models of expertise (Collins, Brown, & Newman, 1987). How could the apprenticeship-style learning from traditional crafts be transferred to the development of cognitive skills such as teaching and curriculum development? One large divide needs to be bridged: the difference between cognitive and traditional apprenticeship is that cognitive apprenticeship focuses on cognitive and metacognitive, not physical, skills. Apprenticeships, argued Collins and his co-authors, exposed the process of creation to apprentices; a young man would see his teacher sanding a cabinet, a young woman would watch her teacher wrap a newborn child. It’s somewhat harder to “see” what goes on inside a faculty member’s head when (for instance) they are reconceptualizing what it means to teach design across the disciplines.

\subsubsection{Techniques used to teach cognitive apprentices}

Since cognitive activity is not visible by default, teachers of intellectual subjects would need to practice “making thinking visible,” or “the externalization of processes that are usually carried out internally... to bring these tacit processes into the open." (Collins, Brown, & Holum, 1991, p. 6) In effect, Collins and his coauthors were saying that facilitating cognitive apprenticeships is all about making one’s metacognition visible to learners in one’s CoP. The several techniques they articulated for doing so are likely to appear in this study’s data.

\emph{Modeling}, where another practitioner performs the task in front of learners so they can see how it is done. Faculty-as-learners have extensive experience with this technique: namely, they have sat as a student in someone else’s classroom for years, and their experiences as a student have a significant impact on how they teach — although not as much as their experiences as teachers (Oleson & Hora, 2012). As faculty, they can engage in classroom observations to watch other faculty teach. In an ideal modeling situation, the practitioner who is modeling also thinks out loud, slows down, and explains intermediate steps: why was a particular decision made in designing this curriculum? What was the practitioner paying attention to while teaching?

\emph{Coaching}, where another practitioner watches the learner perform the task and gives them hints from the sidelines as they attempt it. Peer reviews of teaching, where classroom observations by a fellow faculty or staff member are followed by a feedback session, are one way some faculty already engage in this technique.

\emph{Scaffolding}, where another practitioner helps the learner perform the task, for example by starting it halfway and then letting the learner finish it once the task is at an easier point. Team-teaching with a more experienced faculty member is an example of this.

\emph{Fading} (also called “Exploration”), where learners are encouraged to go out and tackle the process on their own with less and less help from the preceding practices. This is similar to the transition from teaching assistant to independent instructor.

\emph{Articulation}, where a practitioner asks learners to think out loud – an inversion of the “modeling” technique that helps learners become teachers to each other. An example of this is a faculty member stopping in the middle of a lesson and saying “you know what, let’s try something else; everyone is plugging random numbers into their simulation, so let’s stop and see if we can derive the right answer together here on the board.” They are explaining their teaching decisions as they make them and carry them out.

\emph{Reflection}, where learners compare their process to an expert's process or their internal model of a "good" process. When a faculty member looks up to a teaching role model, they are often engaging in this process.

\subsubsection{The goals of cognitive apprenticeships are metacognition and self-monitoring}

Regardless of the specific techniques employed, the ultimate goal of cognitive apprenticeship is the development of metacognition and self-monitoring in students – or as Donald Schön would put it, the ability of reflection-in-action (1983). Developing this ability enables practitioners to serve as “teachers” to one another; faculty can help other faculty develop without being faculty development specialists or even “expert” practitioners of teaching or curriculum design themselves. The only requirement is that the practitioner is self-aware enough to articulate their own performance, including their mistakes. For this reason, I have used the word “practitioner” rather than “teacher” in describing the cognitive apprenticeship techniques, unlike Collins and his colleagues, who were writing for an audience of K-12 teachers and education researchers.

\subsubsection{Limitations of the cognitive apprenticeship perspective: ignoring the agency and experiences of adult learners}

It is here that we find one of the major limitations of the cognitive apprenticeship framework in thinking about faculty-as-learners: it was developed in a context where the learners were children, whereas engineering and technology faculty are adults. Agency is thus implicitly assumed to originate with the “teacher,” eventually flowing outward to the learner. This is evident not just in phrases such as “give students the opportunity” (Collins, Brown, & Newman, 1987, p. 18) and “create a culture… for students” (p. 7), but in the techniques themselves — “modeling” is a predominantly passive act on the part of the learner, and “scaffolding” and “fading” are simply the deliberate and gradual transfer of agency to the learner. The end goal is for the student to have agency, to allow “the role of ‘expert’ and ‘student’ to be transformed,”(Collins, Brown, & Holum, 1991, p. 17) — but it does not start out that way. 

However, Jane Vella’s 1997 book points out that adult learners must be seen as subjects rather than objects in learning from the very beginning (p. 129-148); they are already used to acting as independent agents and bring a rich storehouse of past experiences to the table. Baxter-Magolda and King describe such learners as self-authoring (Baxter-Magolda 7 King, 2004), which means they take responsibility for their own beliefs, sense of self, and relationships with others. In the next section, we will explore another viewpoint that takes the agency of faculty into account.

\subsection{Viewing faculty-as-learners as narrators}

In the preceding section, we explored cognitive apprenticeships as one possible way to see faculty-as-learners. Despite many advantages, this perspective had one significant drawback: because it was originally designed for young learners, its assumptions of agency are limited, as is its ability to draw on the prior experiences of learners.

Another way to see faculty-as-learners is to view them as narrators. Glesne speaks of the writer/narrator role as threefold: (1) artist, (2) translator/interpreter, and (3) transformer (Glesne, 2011, p. 219). The role of narrator can be a high-agency role, with narrators deciding how to create (as artists) a story-telling moment and how to translate it for their audience in order to elicit the desired reaction/transformation. Agency is heightened even more when the narratives are about one's own past. Autobiographical narrators paint themselves as characters in their own stories, drawing from their prior experiences and using their agency in the present to articulate their agency in the past. This "intentional state entailment" is a key feature of narratives; without characters with agency who make choices, we cannot have narratives at all (Bruner, 1991, p. 7).

Narratives give us both a method and methodology for understanding faculty as learners that address the shortfalls of agency and prior-experience we found with the cognitive apprenticeship framework. This subsection will explore narrative analysis as methodology, examining the philosophical perspectives underlying and shaping the method.

\subsubsection{Existing use of narrative methodology in engineering education research}

Just like the framework of cognitive apprenticeship, the use of narrative methodology in engineering education research is not new, and has already been interwoven into CoP theory in that domain. A 2007 paper titled “Storytelling in Engineering Education” highlights the importance of personal narratives as ways to express community values by “[providing] a vehicle for scholarly discourse that makes explicit our implicit knowledge, promotes reflective practice, and provides entry points into a community of practice” (Adams et al., 2007, p. 4-5). Case and Light list “narrative analysis” in their 2011 Journal of Engineering Education paper “Emerging methodologies in engineering education research,” as does the Cambridge Handbook of Engineering Education Research (Johri & Olds, 2014) and Nancy Chism’s booklet Qualitative Research Basics: A Guide for Engineering Educators circulated by the Rigorous Research in Engineering Education (RREE) group (2010). Narrative has been used specifically for research on faculty within engineering education: Pawley’s 2009 paper “Universalized Narratives: Patterns in How Faculty Members Define “Engineering” (2009) and Fincher and Tenenberg’s Disciplinary Commons project which seeks to “[move] narrative from its naturalistic role in teacher conversation to a more purposeful investigation” (Fincher 2012, p. 28) are both examples in this area.

\subsubsection{Narrative is a reflective search for understanding that reexamines our notions of “truth”}

In terms of methodology, narrative work usually falls within the intepretivist paradigm, where the purpose of research is to seek understanding, as opposed to creating predictions or causing emancipation (Glesne, 2011, p. 7). Bruner's landmark 1991 paper, "The Narrative Construction of Reality," removes the boundaries between the mental process of thought and the discourse of its expression as a narrative. The narrative is not sitting precomposed in some idealized platonic space, waiting to be spoken or written by an unthinking scribe. Rather, narrative "operates as an instrument of mind in the construction of reality," (p. 6) and as such, cannot simply be chopped into parts for neat analysis because of its "part-whole textual interdependence."

Looking at the world through a narrative paradigm also requires that we examine our epistemological assumptions about the nature of "knowledge" and "truth." We distinugish between the "constructions generated by logical and scientific procedures that can be weeded out by falsification" and the "version of reality whose acceptability is governed by convention and 'narrative necessity' rather than by empirical verification and logical requiredness" (Bruner, 1991, p. 4-5). The first is called "forensic truth," the second one "narrative truth," a distinction made by the South African Truth and Reconciliation Commission (1998) to recognize that personal recollections of intense memories are not rendered invalid by a lack or confusion of precise dates, quotes, or other “facts.” As a knowledge-seeking methodology, narrative does not seek forensic truth; rather, it seeks verisimilitude, the possibility or resemblance of forensic truth and the truth of personal recollection and memory.

\subsubsection{Narratives are hermeneutic}

Coming from this intensely personal perspective, narratives become boundary objects and mediated dialogues between an author (speaker) and a reader (listener), with both parties taking a highly active role in the process. In Bruner's words, narratives have "hermeneutic composability," meaning that they are things through which people express and extract meaning, but there is no single absolute meaning that can simply be dissected. Two people reading the same book can come out with very different insights. The interpretation being made depends on the background knowledge and intention of both author and reader, as well as what the author and reader know about each other. (Bruner, 1991, p. 7-11) However, the information dissemination model of learning does not account for this hermeneutical interaction.

This hermeneutic composability leads to an expanded perspective on the part of both author and reader. Bruner writes about how narratives have context sensitivity and negotiability. By seeing that we and others may have different contexts, we are able to accept these differences. We recognize that we can immerse ourselves, like anthropologists, into someone else's process for constructing meaning (p. 16-18). Belenky describes the process from the perspective of a constructivist, where participants engage in "...becoming and staying aware of the workings of their minds... [seeking] to stretch the outer boundaries of their consciousness — by making the unconscious conscious, by consulting and listening to the self, by voicing the unsaid, by listening to others and staying alert to all the currents and undercurrents of life about them, by imagining themselves inside the new poem or person or idea that they want to come to know and understand" (Belenky, 1997, p. 141).

\subsubsection{Narratives as communal: the idea of the “narrative accrual”}

Narrative hermeneutics do not just expand perspectives within the author-to-reader connection. In fact, Bruner specifically depicts narratives as communal. Human societies pull multiple narratives into a larger assemblage of many narratives, a "narrative accrual" we share with others of our culture (Bruner, 1991, p. 18-20). The history of a country, the "foundational" papers of an academic discipline, and the dinner-table stories that "everyone in the family knows" are all examples of narrative accruals. These narrative accruals are important enough that we legislate that children to learn national and world history in school, require graduate students to focus their first few years on reading a common core of "foundational" works, and make sure that prospective sons- or daughters-in-law learn certain family stories and traditions when they come to visit. Communities of practice and their narrative accruals therefore co-construct each other.

Learning the stories of one's community by becoming a fluent reader of this narrative accrual is a key part of enculturating into a community of practice. Without a common language, we are unable to communicate. Think about two kids on the playground bonding over a favorite TV show: "Did you see the Star Trek episode ‘Darmok’ where Picard has to figure out how these aliens talk?" "Yeah, and then he realizes the weird words are actually titles of their stories, but he doesn’t know what the stories are..." Alternatively, think about the way researchers refer to common theories to get their ideas across; by invoking Wenger & Lave's communities of practice theory in the preceding section, and Bruner's narrative analysis work here, I draw my research into a web of ideas others have already thought and written about. Practitioners tell each other stories about their work all the time; telling the "right" kinds of stories about the "right" kinds of things (for instance, in a literature review) is a mark of belonging in its own right. "For newcomers," say Lave and Wenger, "the purpose is not to learn from talk as a substitute for legitimate peripheral participation; it is to learn to talk as a key to legitimate peripheral participation" (1991, p. 109).

\subsubsection{When we co-author a community’s narrative accrual, we both enter and transform both the community and ourselves.}

However, it is not enough to become a passive reader of this narrative accrual; a full community member must contribute to the joint process of creating the collective story pool. "Our individual autobiographies... depend on being placed within a continuity provided by a constructed and shared social history in which we locate our Selves and individual continuities. It is a sense of belonging to this canonical past that permits us to form our own narratives of deviation while maintaining complicity with the canon" (Bruner, 1991, p. 20). As we place and shape our own narratives within the narrative accrual of our community, we place and shape ourselves within and in relationship to our community. The transition from passive reader to active writer can be difficult. Marge Piercy’s poem “Unlearning to Not Speak” evokes the learning process of individual-yet-communal sensemaking that engaging in narration can foster: "She must learn again to speak / starting with I / starting with We / starting as the infant does / with her own true hunger" (1987, p. 97). Although narrative is rooted in the “true hunger” of self, it is only in a mutual engagement with the stories of "I" and "We" that sensemaking can begin its work of bridging and transforming.

Just as communities of practice overlap and bridge across each other, so do narrative accruals. A Japanese child my age may have shared my weekly viewing of the Pokemon TV show, but she may have watched the Japanese show Sailor Moon immediately afterwards, whereas I turned off the TV and read American novelists like Mark Twain. These overlaps can and do often interact to cause interesting shifts and merges in the "libraries" of individuals, and eventually in the narrative accruals of a broader culture itself. For instance, two college roommates may talk about their favorite band, then one may introduce the other to a new musician: "If you like Sara Bareilles, you should listen to Vienna Teng." Eventually, if enough people in their social group come to enjoy Vienna Teng, that musician's albums enter their narrative accrual and become a source of lyrics to be quoted, songs to be sung on road trips, and so forth. Similarly, a researcher may start writing from what she thinks will be common ground with her readership ("you're probably already familiar with some cognitive apprenticeship literature and the idea of narrative analysis...") and then branch out into what's likely to be less familiar territory ("...now let me lead you through an exploration of how poststructuralism ties into the ideas we’ve already discussed.")

\subsection{Viewing faculty-as-learners from a poststructural perspective: writerly agents}

Another way of seeing faculty-as-learners is from within the poststructural paradigm, which treats them as agents working against metanarratives by constantly deconstructing intertextual accounts in a writerly manner. Poststructuralism seeks to remain in the chaotic liminal spaces most other paradigms seek to escape, and this study uses tools and language from within the paradigm to help participants and readers stay in the discomfort and “ask questions about what we have not thought to think, about what is most densely invested in our discourse/practices, about what has been muted, repressed, unheard” (Lather, 1991, p. 145).

Since this perspective (and terminology) is unfamiliar to most engineering education researchers, we’ll begin our explanation of poststructuralism in more familiar territory, with Jerome Bruner’s assertion that all narratives are hermeneutically composed (1991, p. 7). This statement from the narrative perspective has a poststructuralist parallel: “All texts are writerly.” 

\subsubsection{Taking a writerly view of narrative accruals empowers faculty-as-learners to remake their stories}

The terms writerly and readerly were coined by Roland Barthes in his 1973 book Le Plaisir du Texte (which the 1975 English translation renders as “The Pleasure of the Text”) in order to refer to the role of a text’s reader. A readerly text treats readers only as readers, passive recipients of knowledge. This may be done through narrative seduction, where authoritative-sounding words make it seem like an author’s interpretation is the only correct one; war propaganda frequently uses this technique. It may also be done through banality, where readers are conditioned to find a text so “ordinary” that they dismiss it with habitual patterns (Bruner, 1991, p. 9-10) such as “Oh, that’s just another woman being oversensitive and emotional.” Mishler’s description of “short-[circuiting] the problem of meaning” by “suppressing the discourse and assuming shared and standard meanings” (1986, p. 65) is a perfect depiction of banality. 

Regardless of the technique used to render a text readerly, such a text places readers in the epistemological position that Baxter-Magolda calls “following formulas” (2001) and Kegan calls the “third order of consciousness” whereby external voices serve as privileged authorities (1994). In a sense, if expertise is the ability to independently create and navigate within a complex context, readerly texts restrict readers from reaching for higher levels of expertise (S. Dreyfus & H. Dreyfus, 1980) by forcing them to play the role of being "'convinced'" of the author's right to serve as 'The Great Interpreter'" (Lather, 1991, p. 10).

Foucault’s 1969 lecture “What is an Author?” and Umberto Eco’s essay collection titled “The Role of the Reader” (1979) speak out against readerly texts and echo Barthes’ call for the death of the author (1967) and the deliberate exposure of the hermeneutic richness inherent in all texts. Texts that do this are writerly texts, and they demand that readers co-construct meaning with the author as they work through the text, much like a workbook with blank spaces for reflective exercises. In a sense, the writerly text is incomplete unless a reader is actively grappling with it, working as a co-author and using their own context to inform a hermeneutics of suspicion. Barthes’ key insight was that all texts are writerly. Cavallaro points out that the very word “text” comes from the Latin “texere,” meaning “to weave.” (2001, p. 59). The fabric of texts can be endlessly made and unmade; they aren't closed, finished, or exclusive to one maker. It is simply that some texts pretend to be readerly and make readers forget their own agency, creating an impoverished view of faculty-as-learners.

The forgetting-of-agency has an unfortunate effect on narrative accruals, the collection of stories that a community shares. If a practitioner is only exposed to readerly texts in their field, they are likely to view their field in a readerly way. However, a practitioner cannot co-author a community’s narrative accrual without first viewing the texts as writerly, and that sort of co-authorship is a key part of becoming a full member of a community of practice. Likewise, placing one’s own personal narrative within the narrative accrual of one’s field is a highly writerly act that helps both with sense-making of identity at a personal level and cultural transformation at the community level. It’s the difference between passively watching a professional basketball game and going home thinking “I’ll never be that good, so why try?” and being inspired by the same game to get out there and play basketball with your friends so that you all improve your game. By making-visible the underlying writerly nature of texts, faculty-as-learners become conscious of how they are already crafting the narratives of their teaching identities. More importantly, they are viewed as learners who recognize and actualize their agency to analyze and edit and shape not only their own stories, but each other’s.

\subsubsection{Rejecting metanarratives, binaries, and the simple notion of “truth” in order to admit multiplicities of perspectives}

The notions of readerly and writerly texts are characteristic of postmodern and poststructural thinking, heavily intertwined movements that share an opposition to positivism and objectivism. Postmodernism is a response to the modernist emphasis on unrelenting “progress” towards a “better” world. By questioning the very concept of a single, infinitely extendable “better” and the promise of certainty that the grand, sweeping metanarrative of continuous improvement offers, postmodernism exemplifies an “incredulity towards metanarratives” (Lyotard, 1984, p. xxiv) that “denies itself the solace of good forms” and “searches for new presentations, not in order to enjoy them but in order to impart a stronger sense of the unpresentable” (p. 81).

The culture of college and university faculty, like all cultures, has metanarratives; they are convenient shorthands used to align activity and thought. One trouble with metanarratives such as “if everybody goes to college, our country would be great!” or “if all professors taught with hands-on projects rather than lectures, college teaching would be excellent!” is that they are totalizing. Metanarratives try to explain everything, but are themselves ahistorical and acultural, typically falling into readerly mode to gloss over their own socially constructed nature and any history of accepted thought that conflicts. “This is the truth,” they say. “This is how it’s always been.” Lyotard argues against totality and for a multiplicity of competing ideas that do not layer neatly atop each other, a view consonant with statements by scholars on the importance of academic freedom and intellectual dissent.

Lyotard’s argument against totality and for multiplicity is a binary (X vs. Y), which is itself rejected by poststructuralism in a tongue-in-cheek binary opposition. Just as postmodernism helps us understand faculty-as-learners by rejecting the modernist notion of a metanarrative of “truth” that they must live within, poststructuralism helps us understand faculty-as-learners by rejecting the “structuralist dream of producing scientific accounts of culture by discovering its underlying sign systems” (Cavallaro, 2001, p. 25), or the objectivism-tinged idea that we can find or co-construct the way things “really work.”

When we try to understand faculty-as-learners from a poststructural perspective, we are not trying to find the “truth” of how they “really are,” as a positivist or post-positivist would. We are not trying to co-construct that truth with them, the way a constructivist might, or “liberate” them into a larger truth, as a critical theorist might. Instead, we simultaneously draw upon and challenge all these intellectual traditions in order to challenge the very idea of a coherent, pragmatic truth. Poststructuralism is a paradigm that deliberately resists convergence. When we try to understand faculty-as-learners from this perspective, we seek to stay with them inside the sort of incoherent liminal spaces where most learners are intensely uncomfortable.

\subsection{Poststructural adaptations of cognitive apprenticeship and narrative frameworks}

Poststructuralists frequently uses other frameworks as components in the collage it stitches together. I will demonstrate this technique by describing poststructural adaptations of the two frameworks we have just discussed: narrative and cognitive apprenticeship. We will then explore how all three frameworks work together to create space for additional theories in the project.

\subsubsection{A poststructural view of narrative}

Poststructuralism does not reject narratives; rather, it embraces them as texts in different ways. The emphasis on narrative agency and the use of personal history still makes narrative valuable as a tool for exploring the idea of faculty-as-learners. However, we no longer see these narratives or the process of creating them as positivist “mirrors” reflecting the “true” world, or as constructivist “lamps” from within the narrator’s mind radiating light on their personal perceptions of “truth” (Abrams, 1953). Both the “mirror” and the “lamp” assume a stable external point of reference by which their “truth” can be calibrated and correlated, but even ordinary language is full of rhetoric and marked by the slippage of meaning (Derrida, 1998). For instance, we tell students to “take a course” that will “move them closer to graduation,” but do not literally mean that a “course” is something to physically take hold of, or that it will bodily convey them to a physical location of “graduation.”

Instead, the poststructuralist view of narrative (and all other) texts is that they are collections of signs that “do not embody specific meanings or concepts” and only “ become meaningful when they are decoded according to cultural conventions.” (Cavallaro, 2001, p. 15-16). For example, take the word “design” as a sign that engineering and technology faculty might use frequently. As a sign, “design” consists of a signifier — in this case, a six-letter English word that starts with ‘d’ — and a signified which the word refers to (Saussure, 1986). In computer science terms, the signifier and signified can be loosely thought of as the pointer and the memory location.

Just as a pointer can be renamed and reallocated in a computer program, the link between signifier and signified is arbitrary, socially constructed, and forever shifting — two faculty members may use the signifier “design” to refer to very different signifieds (loosely, “ideas.”) Even a single faculty member may use the word “design” in more than one sense in the same sentence: “I design my design courses so that students experience design as freedom and creativity, but it’s also important that they be able to design things that meet functional specifications.” It may be tempting to diagnose this as a problem of insufficiently precise language; if we had more words for “design,” we would not need to repeat it to signify slightly different meanings. However, the repetition of the word “design” is entwined with its representation; a sign obtains meaning only when it’s used in multiple contexts, since community adoption is what distinguishes a “nonsense” word from a “real” one. Because of this, when we speak, write, or use any other sign system such as language, we represent — re-present — our signifieds with signifiers that already exist, and the way we choose to do so can be illuminating to examine (Cavallaro, 2001, p. 39).

A poststructuralist approach to faculty narratives is one that frequently interrupts itself to examine itself, and thus expects the role of faculty-in-learner to include a high degree of reflection-in-action (Schön, 1983). When approaching a faculty narrative from a poststructuralist viewpoint, we might discuss their choice of signifiers as a way of becoming more aware of the shifting signifieds we play with in our language: “Why did you choose that word just now? You used another word the last time you told this story.” We might try to trace “meaning” through a sign-system and end up exploring rhetorical slippages in our text as we become aware of our efforts to make things converge into a single “truth”: This latter action is known as deconstruction, the idea Derrida is most famous for articulating.

Deconstruction is not a verb that we as agents execute on a text as object; rather, it is something that we notice the text has always done to itself. (Derrida, 1998; Cavallaro, 2001, p. 26). When we deconstruct a text, we make our world more writerly because we see texts as shifting, changing things; one signifier may refer to many signifieds (as with classifiers), and many signifiers may refer to the same signified (as with synonyms). 

Poststructuralists critique binary oppositions because they "freeze" that shifting and can create false dichotomies that make readers forget they have the writerly power to restructure their world: “Will you pass that student or fail them?” “Neither; I can talk with them about taking an incomplete, substituting other work for the project they didn’t finish, and a whole host of other options.” Lather depicts the goal of deconstruction as “neither unitary wholeness nor dialectical resolution.” Instead, deconstruction seeks to “keep things in process, to disrupt, to keep the system in play, to set up procedures to continuously demystify the realities we create, to fight the tendency for our categories to congeal” (Lather, 1991, p. 13). It tugs at us to stay in the liminal spaces where we are learning.

\subsubsection{A poststructural view of cognitive apprenticeship}

Sign systems such as language — including the language we use to tell the stories that become part of our community’s narrative accruals — have a social dimension, with a multivocal, constantly shifting quality that theorists working with cognitive apprenticeship and community of practice frameworks occasionally give a nod to, but do not typically explore in as much depth as poststructuralists. Communities use sign systems to “[chop] up the continuum of space and time into categories” that then compose their reality. This chopping varies wildly between (and even within) communities; when we marvel at the many words Eskimos fit within our single concept of “snow,” we are noticing their creation of a cultural reality with a different fragmentation than our own (Cavallaro, 2001, p. 17).

When we understand the rhetoric of the community of practice that our cognitive apprenticeships are enculturating us into, we also come to understand the dominant ideology mirrored by that rhetoric. For instance, Greco-Roman ideology valued a rhetoric of clear public oration, whereas Renaissance ideology valued a rhetoric of private communications that danced around a point rather than making it directly; the cultural assumptions of each community are reflected in the ways they privileged certain usages of language (p. 30-31).

Poststructuralists also expose the privilege our communities grant to things that can be signified by signs, pointing to feelings, silences, and other things we deem as “real” but cannot express directly. The teaching practices of cognitive apprenticeships — modeling, coaching, scaffolding, and so forth — are tools that can be used to “impart a stronger sense of the unpresentable” (Lyotard, 1984, p. 81) and enculturate learners into a way of being that cannot be fully articulated with signs. Cognitive apprentices are not just “reading” texts composed of written or spoken words; they are reading the entire world around them all the time (Cavallaro, 2001, p. 49). As learners become more fluent at decoding the signs they are surrounded by, they improve their ability to make sense of a complex context; this ability, as we have mentioned earlier, is also known as mastery (S. Dreyfus & H. Dreyfus, 1980).

A poststructural adaptation of communities of practice sees community members — including apprentice members — as texts, since the people are part of the surrounding sign-system those apprentices work to decode. This draws upon Stanley Fish's idea that “readers are readable” (1980) and Alberto Manguel’s idea that humans are books to be read (1997). In short, readers are produced by their culture as “texts” whose meanings are determined by their communities. In a similar way, texts also become readers because "they expect to be read in ways sanctioned by the community and thus read us as we read them, i.e. monitor our ability to employ the interpretive skills we are supposed to have developed" (Cavallaro, 2001). A common example in academia is the difficult “classic” paper that graduate students are expected to read “in ways sanctioned by the community” until they can use it to display the “interpretative skills” they “are supposed to have developed.” We transform writerly texts by reading them; texts transform us by reading us “as we read them.”

\subsubsection{A poststructural view of narrative accruals in a cognitive apprenticeship}

Having introduced the ideas of “readers as texts” and “texts as readers,” we can now describe faculty as learners who simultaneously negotiate both roles. From this perspective, faculty-as-learners can be seen as readers/texts that constantly and mutually re-author and re-present each other, themselves, and the narrative accruals of their communities. This is a high-agency role that requires the faculty narrator/apprentice to draw extensively on both their present context and their past history in order to expose and work within a writerly world. It is a stark contrast to the “information dissemination” view of faculty-as-learners as mere recipients and measurable end-objects in someone else’s project.

“Mutually re-authoring and re-presenting” may sound like a more complicated version of the phrase “they co-construct each other,” and, in a sense, it is. However, the sign “co-construction” usually signifies an attempt to converge, even temporarily, on a single viewpoint. Even statements of the type that “there are many viewpoints; historically, some people have viewed this as X, others as Y, and yet others as Z” is itself based on a single viewpoint that is known on Wikipedia as neutral point of view (NPOV). Its rhetoric of claiming to represent "fairly, proportionately, and, as far as possible, without bias, all of the significant views that have been published by reliable sources on a topic,” mirrors a certain unspoken ideology of what qualifies as a “significant view” or “reliable source” (Wikipedia, 2014). In contrast, the poststructural viewpoint casts faculty narrator/apprentices and their narratives as “polyphonic texts that challenge dominant ideologies by articulating diverse discourses… thus resisting the notion of a unified viewpoint” (Cavallaro, 2001, p. 19).

The resulting accrual of both narrations and narrators also displays intertextuality, the idea that texts “always absorb and transform other texts” and “can be thought of as a tapestry of quotations, a mosaic of allusions” (p. 60). One famous example of intertextuality in action is T.S. Eliot’s poem “The Waste Land” (Eliot, 2000), which consists largely of allusions to other literary works. Another example of intertextuality is a scholarly document such as this one. When I quote Cavallaro as saying that “if texts are intertextual, subjective responses to texts are intersubjective,” I use her signifier of “intersubjectivity” in order to contextualize my thoughts (1991, p. 60).

Since the sign “intersubjectivity” is likely to be unfamiliar to many of my readers, I must employ repetition to re-present the concept in greater detail so that my readers are able to make sense of the new word. Intersubjectivity refers to the interactions between interpretations proposed by multiple people. Intertextuality is about the interweaving of texts in the same way intersubjectivity is about the interweaving of interpretations. For instance, a conversation between faculty members reflecting on a shared experience is intersubjective; although their interpretations of the experience may not ultimately converge, their interpretations will interact and change during the course of the discussion. To draw an even more specific example, the responses of a committee to a dissertation proposal are intersubjective. Each committee member reading the proposal goes through the process of interpreting the document for themselves, but their interpretation interacts with the interpretations of the other committee members and the graduate student author. The interaction of these interpretations takes place within the conventions of a particular community of practice — in this case, the community of engineering education researchers. Based on these culturally-shaped interactions, the interpretations of particular community members may be accepted or rejected.

By defining intersubjectivity, I am engaging in a reflexive interruption and deconstruction of my own text, revealing my own authorship, and encouraging the reader (you!) to see this document as a writerly text to freely interpret and make sense of. None of the ideas introduced in this section of the document are particularly new; what I’m trying to do is to give you a language you can use to discuss and become more aware of how these dynamics already play out in the lives of faculty. All the same, the process of acquiring poststructural language is often an uncomfortable process for the reader, who may not be used to authors breaking the fourth wall. For this discomfort, I apologize. However, this boundary-blurring is typical of the poststructural approach of “den[ying]… the solace of good forms” (Lyotard, 1984, p. 81) in order to get readers to examine their expectations of how texts ought to perform.
